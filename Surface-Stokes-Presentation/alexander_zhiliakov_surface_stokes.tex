\documentclass[svgnames]{beamer} % for xcolor https://latex.org/forum/viewtopic.php?t=2445 
\mode<presentation>
{
\usetheme{Warsaw}
\setbeamertemplate{page number in head/foot}[totalframenumber]
%\setbeamertemplate{footline}[frame number]
\setbeamertemplate{headline}{}
\setbeamercovered{transparent}
% or whatever (possibly just delete it)
}

\usepackage[english]{babel}
\usepackage[utf8]{inputenc}

\title[Trace FEM]{Unfitted finite element methods for fluid flows on surfaces}
\institute[UH] {
	\includegraphicsw[.3]{logo_uh.png}
}
\date[October 17, 2020]{SIAM CSE 2021, March 1, 2021}

% bold for everything
\usepackage{bm}
% lowercase mathcal font
\usepackage{dutchcal}
\hypersetup{
	colorlinks,
	allcolors=.,
	urlcolor=blue,
	filecolor=blue
}
% braces for subeqns and boxes
\usepackage{empheq}
% http://mirror.hmc.edu/ctan/macros/latex/contrib/mathtools/empheq.pdf
\newcommand*\widefbox[1]{\fbox{\hspace{1em}#1\hspace{1em}}}
% hl
\usepackage{soul}
\makeatletter
\let\HL\hl
\renewcommand\hl{%
	\let\set@color\beamerorig@set@color
	\let\reset@color\beamerorig@reset@color
	\HL}
\makeatother
% sub figures / grids of pictures
\usepackage{overpic}
\usepackage{subcaption} 
\graphicspath{{./img/}} % includegraphics path
\newcommand{\includegraphicsw}[2][1.]{\includegraphics[width=#1\linewidth]{#2}}
\newcommand{\svginput}[1]{\input{img/#1}} % pdf_tex path
\newcommand{\svginputw}[2][1.]{\def\svgwidth{#1\linewidth}\input{./img/#2}} % pdf_tex path
% tables
\let\oldtabular\tabular
\renewcommand{\tabular}[1][1.5]{\def\arraystretch{#1}\oldtabular}
\usepackage{hhline}
\usepackage{multirow}
% \coloneqq
\usepackage{mathtools}
% math commands for convinience
\DeclareMathOperator{\argmin}{arg\,min}
% bold vectors
\newcommand{\vect}[1]{\boldsymbol{\mathbf{#1}}}

% differentials
\newcommand*\diff{\mathop{}\!\mathrm{d}}
\newcommand*\Diff[1]{\mathop{}\!\mathrm{d^#1}}

\let\pa\partial
\let\ep\varepsilon
\newcommand{\R}{\mathbb R}
\newcommand{\bA}{\mathbf A}
\newcommand{\bB}{\mathbf B}
\newcommand{\bC}{\mathbf C}
\newcommand{\bD}{\mathbf D}
\newcommand{\bE}{\mathbf E}
\newcommand{\bH}{\mathbf H}
\newcommand{\bI}{\mathbf I}
\newcommand{\bJ}{\mathbf J}
\newcommand{\bP}{\mathbf P}
\newcommand{\bK}{\mathbf K}
\newcommand{\bM}{\mathbf M}
\newcommand{\bQ}{\mathbf Q}
\newcommand{\bS}{\mathbf S}
\newcommand{\cV}{\mathcal{V}_{\rm div}}
\newcommand{\bV}{\mathbf V}
\newcommand{\ga}{\gamma}
\newcommand{\ba}{\mathbf a}
\newcommand{\bb}{\mathbf b}
\newcommand{\bg}{\mathbf g}
\newcommand{\blf}{\mathbf f}
\newcommand{\bn}{\mathbf n}
\newcommand{\be}{\mathbf e}
\newcommand{\bp}{\mathbf p}
\newcommand{\br}{\mathbf r}
\newcommand{\bs}{\mathbf s}
\newcommand{\bt}{\mathbf t}
\newcommand{\bT}{\mathbf T}
\newcommand{\bu}{\mathbf u}
\newcommand{\bU}{\mathbf U}
\newcommand{\bv}{\mathbf v}
\newcommand{\bw}{\mathbf w}
\newcommand{\bW}{\mathbf W}
\newcommand{\bx}{\mathbf x}
\newcommand{\by}{\mathbf y}
\newcommand{\bz}{\mathbf z}
\newcommand{\bbf}{\mathbf f}
\newcommand{\A}{\mathcal A}
\newcommand{\F}{\mathcal F}
\newcommand{\G}{\mathcal G}
\newcommand{\I}{\mathcal I}
\newcommand{\K}{\mathcal K}
\newcommand{\M}{\mathcal M}
\newcommand{\N}{\mathcal N}
\newcommand{\Q}{\mathcal Q}
\newcommand{\J}{\mathcal J}
\newcommand{\T}{\mathcal T}
\newcommand{\V}{\mathcal V}
\newcommand{\Div}{\mbox{div}}
\newcommand{\divG}{{\mathop{\,\rm div}}_{\Gamma}}
\newcommand{\gradG}{\nabla_{\Gamma}}
\newcommand{\nablaG}{\nabla_{\Gamma}}
\newcommand{\cH}{\mathcal H}
\newcommand{\cD}{\mathcal D}
\newcommand{\cF}{\mathcal F}
\newcommand{\cE}{\mathcal E}
\newcommand{\cO}{\mathcal O}
\newcommand{\cT}{\mathcal T}
\newcommand{\cP}{\mathcal P}
\newcommand{\cR}{\mathcal R}
\newcommand{\cS}{\mathcal S}
\newcommand{\cL}{\mathcal L}
\newcommand{\rr}{\mathbb{R}}
\newcommand{\cc}{\mathbb{C}}
\newcommand{\kk}{\mathbb{K}}
\newcommand{\nn}{\mathbb{N}}
\newcommand{\dd}{\,d}
\newcommand{\zz}{\mathbb{Z}}
\newcommand{\Gs}{\mathcal{S}} %{\Gamma_\ast}
\newcommand{\OGamma}{\Omega^\Gamma_h}
\newcommand{\tA}{\widetilde{\mathcal A}}
%\newcommand{\Gsn}{{\Gamma_\ast^n}}
%\newcommand{\normal}{n}
\newcommand{\wn}{w_{\!\perp}}
\newcommand{\ddp}{\partial}
\newcommand{\ati}{\tilde{a}}
\newcommand{\wtang}{\bw_\parallel}
%\newcommand{\grad}{\nabla}
\newcommand{\jumpleft}{[\![}
\newcommand{\jumpright}{]\!]}
\newcommand{\averageleft}{\{\!\!\{}
\newcommand{\averageright}{\}\!\!\}}
\newcommand{\divergence}{\textrm{div}ARJankuhnonvectroLaplace\ \!}
\renewcommand{\div}{\mbox{\rm div}\ \!}
\newcommand{\HSobolev}{H}
\newcommand{\Lzwei}{L^2}
\newcommand{\Rn}{\mathbb{R}^n}
\newcommand{\Rm}{\mathbb{R}^m}
\newcommand{\tr}{{\rm tr}}
%\newcommand{\alert}[1]{{\bf #1}}
%\DeclareGraphicsExtensions{.pdf,.eps,.ps,.eps.gz,.ps.gz,.eps.Y}
\def\Fh{\mathcal{F}_h}
\def\Eh{\mathcal{E}_h}
\newcommand{\la}{\left\langle}
\newcommand{\ra}{\right\rangle}
\newcommand{\Yo}{\overset{\circ}{Y}}
\newcommand{\deriv}[2]{\frac{\partial #1}{\partial #2}}
\newcommand{\bsigma}{\boldsymbol{\sigma}}
\newcommand{\DG}{D_\Gamma}
\newcommand{\DGh}{D_{\Gamma_h}}
\newcommand{\bxi}{\mbox{\boldmath$\xi$\unboldmath}}
\def\cl {\nonumber \\}
\def\el {\nonumber }
\newcommand{\lPlane}{L_{\vect x_0}}
\newcommand{\sphere}{{\Gamma_{\text{sph}}}}
\newcommand{\tor}{{\Gamma_{\text{tor}}}}
% for norms
\newcommand{\HOne}{{H^1}}
\newcommand{\LTwo}{{L^2}}
\newcommand{\LTwoSpace}[1][\Gamma]{{L^2\left({#1}\right)}}
\newcommand{\HOneSpace}[1][\Gamma]{{H^1\left({#1}\right)}}

\begin{document}

	\author[Alex Zhiliakov]{%
		\small\begin{tabular}[1.]{cl}
			\textbf{Alexander Zhiliakov} & \qquad Collaborators: \\
			\href{mailto:alex@math.uh.edu}{alex@math.uh.edu} & \qquad Maxim Olshanskii (UH) \\
			& \qquad Arnold Reusken (RWTH Aachen) \\
			& \qquad Thomas Jankuhn (RWTH Aachen)
		\end{tabular}
		\vskip -1mm
	}

	\begin{frame}
		\titlepage
	\end{frame}

	\begin{frame}{Overview}
		\tableofcontents
	\end{frame}

	\section{Continuous formulation}
	
	\begin{frame}{Continuous formulation}
		\vskip -.5cm
		\begin{align*}
			-2\,\bP\,\div_\Gamma (E_s(\bu)) + \alpha\,\bu + \nabla_\Gamma p &=  \blf \: \text{on}~\Gamma, \quad 	E_s(\bu) \coloneqq \frac 12\left(\nabla_\Gamma\bu + \nabla^T_\Gamma\bu \right), \\
			\div_\Gamma \bu & = g \:\text{on}~\Gamma,\quad
			{\bu\cdot\bn}=0 \:\text{on}~\Gamma
		\end{align*}
		\vskip -.2cm
		\begin{block}{Weak formulation}
			Find %$(\bv_T,p) \in \bV_T^0 \times L_0^2(\Gamma)$ (for $\alpha=0$) or
			$(\bu,p) \in \bV_T \times L_0^2(\Gamma)$ such that
			\vskip -.2cm
			\begin{equation}\label{cont}
			\begin{aligned}
			a(\bu,\bv) +b_T(\bv,p) &=(\blf,\bv), \quad\quad \bv \in \bV_T, \\
			b_T(\bu,q) & = (-g,q), \quad q \in L^2(\Gamma)
			\end{aligned}
			\end{equation}
		\end{block}
		\only<1>{%
			\begin{align*}
				a(\bu,\bv) &\coloneqq \int_\Gamma (2\,E_s(\bu):E_s(\bv)+\alpha\,\bu\cdot\bv) \diff{s},\\
				b_T(\bv,p) &\coloneqq -\int_\Gamma p\,\divG \bv_T \diff{s}, \quad \bV_T\coloneqq  \{\, \bv \in H^1(\Gamma)^3\::\: \bv\cdot \bn =0\,\}
			\end{align*}
		}%
		\only<2>{%
			Problem~\eqref{cont} is well-posed thanks to Korn inequality and inf-sup property~[1]
			$$
				\|\bv_T\|_{L^2(\Gamma)}+ \|E_s(\bv_T)\|_{L^2(\Gamma)} \geq c \|\bv_T\|_{1},\quad
				\sup_{\bv_T\in{\bV_T}}\frac{b_T(\bv_T,p)}{\|\bv_T\|_{1}}  \geq c \|p\|_{L^2(\Gamma)}
			$$
			\setul{1pt}{.4pt} % https://tex.stackexchange.com/questions/50637/closer-underline
			\tiny{
				\textsuperscript{1}T.~Jankuhn, M.~Olshanskii, A.~Reusken\\
				$\quad\:\:$\href{https://arxiv.org/abs/1702.02989}{\ul{Incompressible fluid problems on embedded surfaces: Modeling and variational formulations}},\\
				$\quad\:\:$Interfaces and Free Boundaries 2018
			}
		}
	\end{frame}
	
	\section{Discrete formulation}
	
	\begin{frame}{Discrete formulation}
		\begin{block}{Discrete weak formulation}
			Find $(\bu_h, p_h) \in \bU_h \times Q_h$ such that
			\vskip -.2cm
			\begin{equation}\label{discrete_}
				\begin{aligned}
				A_h(\bu_h,\bv_h) + b_T(\bv_h,p_h) & =(\blf,\bv_h), &\quad &\bv_h \in \bU_h,\\
				b_T(\bu_h,q_h)-s_h(p_h,q_h) & = (-g,q_h), &\quad &q_h \in Q_h
			\end{aligned}
			\end{equation}
		\end{block}
		\begin{figure}
			\begin{subfigure}{.6\linewidth}
				\small We include additional terms to enforce \textcolor{Red}{tangentiality} and \textcolor{Green}{algebraic stability}:
				\begin{align*}
				A_h(\bu, \bv) & \coloneqq
				a_h(\bu,\bv) + \textcolor{Red}{\tau \int_{\Gamma_h} u_N\,v_N \diff{s}} \\
				&+\textcolor{Green}{\rho_u \int_{\Omega_h^{\Gamma}} ([\nabla\bu]\,\bn)\cdot([\nabla\bv]\,\bn) \diff{\vect x}},\\
				s_h(p,q) &\coloneqq  \textcolor{Green}{\rho_p  \int_{\OGamma} (\bn \cdot \nabla p)  (\bn \cdot\nabla q) \,  \diff{\vect x}},\\
				\tau &= c_\tau h^{-2},\quad \rho_p= c_p h, \quad \rho_u\in [c_u h,C_u h^{-1}]
				\end{align*}
			\end{subfigure}%
			\begin{subfigure}{.4\linewidth}\centering
				\begin{overpic}[width=0.9\textwidth]{./img/mesh3.png}
					\put(-38,80){\scriptsize Bulk mesh $\longrightarrow$}
					\put(-5,50){\scriptsize $\Gamma_h \longrightarrow$}
					\put(40,105){\scriptsize $\OGamma$}
					\put(40,90){$\Big \downarrow$}
				\end{overpic}
			\end{subfigure}
		\end{figure}
	\end{frame}
	
	\subsection{Inf-sup stability}

	\begin{frame}{Well-posedness}
		\setcounter{equation}{1}		
		\begin{equation}\label{discrete}
		\begin{aligned}
			A_h(\bu_h,\bv_h) + b_T(\bv_h,p_h) & =(\blf,\bv_h), &\quad &\bv_h \in \bU_h,\\
			b_T(\bu_h,q_h)-s_h(p_h,q_h) & = (-g,q_h), &\quad &q_h \in Q_h
		\end{aligned}
		\end{equation}
		\begin{itemize}
			\item Forms $b_T(\cdot,\cdot)$ and $s_h(\cdot,\cdot)$ are continuous and
			the form $A_h(\cdot, \cdot)$ is both coercive and continuous
			\item If we have
			%\begin{exampleblock}{}
			\begin{equation}\label{LBB}
				\|q\|_h \lesssim \sup_{\bv\in\bU_h}\frac{b_T(\bv,q)}{\|\bv\|_A} +s_h(q,q)^{\frac12}, \quad q \in Q_h, \tag{LBB}
			\end{equation}
			%\end{exampleblock}
			then~\eqref{discrete} is well-posed  in the product norm $(\|\cdot\|_A^2+\|\cdot\|_h^2)^{\frac12}$,
			\begin{equation*}
				\|\bv\|_A=A_h(\bv, \bv)^{\frac12},\quad\|q\|_h=\left(\|q\|^2_{L^2(\Gamma)}+s_h(q,q)\right)^\frac12
			\end{equation*}
		\end{itemize}
	\end{frame}
	
	\begin{frame}{Inf-sup stability proof outline\textsuperscript{2}}		
		\begin{equation}\label{LBB_}
			\|q\|_h \lesssim \sup_{\bv\in\bU_h}\frac{b_T(\bv,q)}{\|\bv\|_A} +s_h(q,q)^{\frac12}, \quad q \in Q_h \tag{LBB}
		\end{equation}
		\vfill
		\only<1>{%
			\begin{block}{Step 1: Verf\"urth's trick}
				We show that \eqref{LBB_}~$\Leftrightarrow$
				\begin{equation*}
					|q|_{1,\,h} \coloneqq \Big(\sum_{T \in \T_h^\Gamma} h_T \|\nabla q\|_{L^2(T)}^2\Big)^{\frac12} \lesssim \sup_{\bv\in\bU_h}\frac{b_T(\bv,q)}{\|\bv\|_A} +s_h(q,q)^{\frac12}
				\end{equation*}
			\end{block}
		}%
		\only<2>{%
			\begin{block}{Step 2: Regular elements estimate}
				For~$\cT^{\Gamma}_{\textnormal{reg}} \coloneqq \{\, T \in \cT_h^\Gamma\::\:|\Gamma_T| \geq c\,h_T^2\,\}$ we have%~\eqref{LBB_}~$\Leftrightarrow$
				\begin{align*}
					|q|_{1,\,h}^2 &\lesssim \sum_{T \in \cT^{\Gamma}_{\textnormal{reg}}} h_T \|\nabla q\|_{L^2(T)}^2 +s_h(q,q)\\
					&=\|q\|_{1,\,\textnormal{reg}}^2 +s_h(q,q)
					%|q|_{1,\,\textnormal{reg}} \coloneqq \Big(\sum_{T \in \cT^{\Gamma}_{\textnormal{reg}}} h_T \|\nabla q\|_{L^2(T)}^2\Big)^{\frac12} \lesssim \sup_{\bv\in\bU_h}\frac{b_T(\bv,q)}{\|\bv\|_A} +s_h(q,q)^{\frac12}
				\end{align*}
			\end{block}
		}%
		\only<3>{%
			\begin{block}{Step 3}
				Estimating
				\begin{align*}
					|q|_{1,\,\textnormal{reg}} &\lesssim \frac{ b_T(\bv,q)}{ |q|_{1,\,\textnormal{reg}}+s_h(q,q)^\frac12} +s_h(q,q)^\frac12, \\
					 \|\bv\|_A &\lesssim |q|_{1,\,\textnormal{reg}}+s_h(q,q)^\frac12,
				\end{align*}
				we get~\eqref{LBB_}
			\end{block}
			Higher order case~$\vect P_k$--$P_{k-1}$, $k > 2$, analyzed and proved in~[3]
		}%
		\vfill
		\setul{1pt}{.4pt} % https://tex.stackexchange.com/questions/50637/closer-underline
		\tiny{
			\textsuperscript{2}M.~Olshanskii, A.~Reusken, A.~Zhiliakov\\
			$\:\:$\href{https://arxiv.org/abs/1909.02990}{\ul{Inf-sup stability of the trace $\vect P_2$--$P_1$ Taylor--Hood elements for surface PDEs}}, Mathematics of Computation 2020
		}
		\only<3>{\tiny{
			\textsuperscript{3}T.~Jankuhn, M.~Olshanskii, A.~Reusken, A.~Zhiliakov\\
			$\:\:$\href{https://arxiv.org/abs/2003.06972}{\ul{Error analysis of higher order trace finite element methods for the surface Stokes equation}},\\
			$\:\:$Journal of Numerical Mathematics 2020
		}}
	\end{frame}
	
	\subsection{Error bounds}
	
	\begin{frame}{Error estimates}
		Employing inf-sup stability, coercivity of $A_h(\cdot, \cdot)$, Galerkin orthogonality, and interpolation estimates for $\vect P_2$ and $P_1$ trace finite elements~[4], we have
		\begin{align*}
		\|\bu_T^\ast  -(\bu_h)_T\|_{H^1(\Gamma)}  + \|p^\ast  - p_h \|_{L^2(\Gamma)}  &\lesssim h^2 (\|\bu_T^\ast\|_{H^3(\Gamma)} +\|p^\ast\|_{H^2(\Gamma)}), \\
		\|\bu_h\cdot\bn \|_{L^2(\Gamma)}  &\lesssim h^3 (\|\bu_T^\ast\|_{H^3(\Gamma)} +\|p^\ast\|_{H^2(\Gamma)})
		\end{align*}
		Duality argument yields the optimal error bound in the surface $L^2$ norm for velocity
		\begin{equation*}
		\|\bu_T^\ast  - (\bu_h)_T\|_{L^2(\Gamma)} \lesssim h^{3}(\|\bu_T^\ast\|_{H^3(\Gamma)} +\|p^\ast\|_{H^2(\Gamma)})
		\end{equation*}
		\vfill
		\tiny{
			\textsuperscript{4}M.~Olshanskii, A.~Quaini, A.~Reusken, V.~Yushutin\\
			$\:\:$\href{https://arxiv.org/abs/1801.06589}{\ul{A finite element method for the surface Stokes problem}}, SISC 2018
		}
	\end{frame}
	
	\section{Numerical experiments}
	
	\subsection{Inf-sup constant}
	
	\begin{frame}{Inf-sup constant: independence of~$h$}
		Smallest generalized nonzero eigenvalue~$\lambda_2$ of the pressure Schur complement~$\vect S = \vect B\,\vect A^{-1}\,\vect B^{T} + \vect C$ defines optimal constant~$c^2 = \lambda_2$ in~\eqref{LBB}
		\begin{figure}
			\begin{subfigure}{.3\linewidth}
				\includegraphicsw[.9]{{lvl2.cropped}.png}
				\includegraphicsw[.9]{{tor_lvl3.cropped}.png}
			\end{subfigure}%
			\begin{subfigure}{.7\linewidth}\centering
				%\begin{table}
					\centering\tiny
					\begin{tabular}[1.2]{|c|c|c|c|c|}
						\hline
						\multirow{2}{*}{$h$} & \multicolumn{2}{c|}{$\vect S_0$ (no stab.)} & \multicolumn{2}{c|}{$\vect S_n$ (normal stab.)} \\
						\cline{2-5}
						& $\lambda_2$ & $\lambda_{m}$ & $\lambda_2$ & $\lambda_{m}$ \\
						\hline
						$8.33\times	10^{-1}$	&	$2.33\times	10^{-1}$	&	$1.07$	&	$6.3\times	10^{-1}$	&	$1.$	\\ \hline
						$4.17\times	10^{-1}$	&	$4.72\times	10^{-2}$	&	$6.97\times	10^{-1}$	&	$5.29\times	10^{-1}$	&	$1.$	\\ \hline
						$2.08\times	10^{-1}$	&	$7.93\times	10^{-2}$	&	$6.7\times	10^{-1}$	&	$5.09\times	10^{-1}$	&	$1.$	\\ \hline
						$1.04\times	10^{-1}$	&	$3.71\times	10^{-2}$	&	$6.69\times	10^{-1}$	&	$5.03\times	10^{-1}$	&	$1.$	\\ \hline
						$5.21\times	10^{-2}$	&	$1.81\times	10^{-3}$	&	$6.68\times	10^{-1}$	&	$4.98\times	10^{-1}$	&	$1.$	\\ \hline
						$2.6\times	10^{-2}$	&	$6.65\times	10^{-4}$	&	$6.65\times	10^{-1}$	&	$4.92\times	10^{-1}$	&	$1.$	\\ \hline
					\end{tabular}
					\vskip .5cm
				%\end{table}
				%\begin{table}
					\centering\tiny
					\begin{tabular}[1.2]{|c|c|c|c|c|}
						\hline
						\multirow{2}{*}{$h$} & \multicolumn{2}{c|}{$\vect S_0$} & \multicolumn{2}{c|}{$\vect S_n$} \\
						\cline{2-5}
						& $\lambda_2$ & $\lambda_{m}$ & $\lambda_2$ & $\lambda_{m}$ \\
						\hline
						$2.08\times	10^{-1}$	&	$2.15\times	10^{-1}$	&	$9.56\times	10^{-1}$	&	$3.12\times	10^{-1}$	&	$1.$	\\ \hline
						$1.04\times	10^{-1}$	&	$1.59\times	10^{-2}$	&	$7.6\times	10^{-1}$	&	$3.21\times	10^{-1}$	&	$1.$	\\ \hline
						$5.21\times	10^{-2}$	&	$1.31\times	10^{-3}$	&	$7.48\times	10^{-1}$	&	$3.21\times	10^{-1}$	&	$1.$	\\ \hline
						$2.6\times	10^{-2}$	&	$1.9\times	10^{-4}$	&	$7.42\times	10^{-1}$	&	$3.2\times	10^{-1}$	&	$1.$	\\ \hline
					\end{tabular}
				%\end{table}
			\end{subfigure}
		\end{figure}
	\end{frame}
	
	\begin{frame}{Inf-sup constant: $\Gamma$ and the background mesh}
		We illustrate that the optimal constant in \eqref{LBB} is  uniformly bounded w.r.t. the position of $\Gamma$ in  the background mesh. We consider shifted surfaces
		\begin{equation*}
			\Gamma \mapsto \Gamma + \alpha\,\vect s
		\end{equation*}
		with some $\alpha \in \mathbb R$ and $\vect s \in \mathbb R^3$, $\|\vect s\| = 1$
		\begin{figure}
			\centering
			\begin{subfigure}{.4\linewidth}
				\centering
				\includegraphicsw[.9]{{shift_0.0.cropped}.png}
				\caption{$\sphere$}
			\end{subfigure}%
			\begin{subfigure}{.4\linewidth}
				\centering
				\includegraphicsw[.9]{{shift_0.4.cropped}.png}
				\caption{$\sphere + \alpha\,\vect s$}
			\end{subfigure}%
		\end{figure}
	\end{frame}

	\begin{frame}{Inf-sup constant: $\Gamma$ and the background mesh}	
		\begin{table}
			\centering\only<1>{%
			Sphere, $\Gamma = \sphere$\vskip .3cm
			\begin{tabular}[1.2]{|c|c|c|c|c|}
				\hline
				\multirow{2}{*}{Surface} & \multicolumn{2}{c|}{$\vect S_0$} & \multicolumn{2}{c|}{$\vect S_n$} \\
				\cline{2-5}
				& $\lambda_2$ & $\lambda_{m}$ & $\lambda_2$ & $\lambda_{m}$ \\
				\hline
				$\sphere + 0.0\,\vect s$	&	$3.71\times	10^{-2}$	&	$6.69\times	10^{-1}$	&	$5.03\times	10^{-1}$	&	$1.$	\\ \hline
				$\sphere + 0.1\,\vect s$	&	$1.31\times	10^{-3}$	&	$6.87\times	10^{-1}$	&	$5.03\times	10^{-1}$	&	$1.$	\\ \hline
				$\sphere + 0.2\,\vect s$	&	$1.25\times	10^{-3}$	&	$6.70\times	10^{-1}$	&	$5.03\times	10^{-1}$	&	$1.$	\\ \hline
				$\sphere + 0.3\,\vect s$	&	$1.04\times	10^{-2}$	&	$6.72\times	10^{-1}$	&	$5.03\times	10^{-1}$	&	$1.$	\\ \hline
				$\sphere + 0.4\,\vect s$	&	$5.32\times	10^{-4}$	&	$6.72\times	10^{-1}$	&	$5.03\times	10^{-1}$	&	$1.$	\\ \hline
			\end{tabular}}\only<2>{%
			Torus, $\Gamma = \tor$\vskip .3cm
			\begin{tabular}[1.2]{|c|c|c|c|c|}
				\hline
				\multirow{2}{*}{Surface} & \multicolumn{2}{c|}{$\vect S_0$} & \multicolumn{2}{c|}{$\vect S_n$} \\
				\cline{2-5}
				& $\lambda_2$ & $\lambda_{m}$ & $\lambda_2$ & $\lambda_{m}$ \\
				\hline
				$\tor + 0.00\,\vect s$	&	$1.59\times	10^{-2}$	&	$7.6\times	10^{-1}$	&	$3.21\times	10^{-1}$	&	$1.$	\\ \hline
				$\tor + 0.05\,\vect s$	&	$9.20\times	10^{-3}$	&	$1.14$	&	$3.21\times	10^{-1}$	&	$1.$	\\ \hline
				$\tor + 0.10\,\vect s$	&	$3.00\times	10^{-3}$	&	$1.91$	&	$3.19\times	10^{-1}$	&	$1.$	\\ \hline
				$\tor + 0.15\,\vect s$	&	$8.67\times	10^{-3}$	&	$1.02$	&	$3.21\times	10^{-1}$	&	$1.$	\\ \hline
				$\tor + 0.20\,\vect s$	&	$6.68\times	10^{-3}$	&	$3.04$	&	$3.21\times	10^{-1}$	&	$1.$	\\ \hline 
			\end{tabular}}%
		\end{table}
	\end{frame}
	
	\subsection{Convergence test}
	
	\begin{frame}{Convergence test}
		\begin{figure}[h]
			\centering
			\begin{subfigure}{.4\linewidth}
				\centering
				\includegraphicsw[.8]{u_soln.png}
				\caption{$\vect u(\vect x) \coloneqq \vect P\,(-z^2, y, x)^T$}
			\end{subfigure}%
			\begin{subfigure}{.4\linewidth}
				\centering
				\includegraphicsw[.8]{p_soln.png}
				\caption{$p(\vect x) \coloneqq x\,y^2 + z$}
			\end{subfigure}%
		\end{figure}%
		\vskip -.25cm
		\begin{table}\centering\small
			\begin{tabular}[1.2]{|c||c|c||c|c||c||c|}
				\hline
				Level & $\|\vect u - \vect u_h\|_{\HOne}$ & Order & $\|\vect u - \vect u_h\|_{\LTwo}$ & Order & $\|p - p_h\|_{\LTwo}$ & Order \\
				\hline
				1	&	$2.2$	&	$\text{}$	&	$6.4\times	10^{-1}$	&	$\text{}$	&	$7.4\times	10^{-1}$	&	$\text{}$	\\ \hline
				2	&	$3.8\times	10^{-1}$	&	$2.5$	&	$6.1\times	10^{-2}$	&	$3.4$	&	$1.2\times	10^{-1}$	&	$2.6$	\\ \hline
				3	&	$9.2\times	10^{-2}$	&	$2.1$	&	$5.8\times	10^{-3}$	&	$3.4$	&	$2.5\times	10^{-2}$	&	$2.2$	\\ \hline
				4	&	$2.2\times	10^{-2}$	&	$2.1$	&	$5.6\times	10^{-4}$	&	$3.4$	&	$6.1\times	10^{-3}$	&	$2.1$	\\ \hline
				5	&	$5.3\times	10^{-3}$	&	$2.$	&	$5.2\times	10^{-5}$	&	$3.4$	&	$1.6\times	10^{-3}$	&	$1.9$	\\ \hline
				6	&	$1.3\times	10^{-3}$	&	$2.$	&	$5.2\times	10^{-6}$	&	$3.3$	&	$4.1\times	10^{-4}$	&	$2.$	\\ \hline
				7	&	$3.4\times	10^{-4}$	&	$2.$	&	$6.\times	10^{-7}$	&	$3.1$	&	$1.\times	10^{-4}$	&	$2.$	\\ \hline
			\end{tabular}
		\end{table}
	\end{frame}
	
	\subsection{Kelvin--Helmholtz instability}
	
	\begin{frame}{Kelvin--Helmholtz instability (surface NS)}
		\begin{figure}
			\small{Mixing layer of incompressible viscous flow at $Re = 10^4$}
			\begin{subfigure}{.5\linewidth}
				\centering
				\href{file:./img/KH.avi}{\includegraphicsw[.7]{kh_ini_u.png}}
			\end{subfigure}%
			\begin{subfigure}{.5\linewidth}
				\includegraphicsw[.9]{kh_kinetic_energy.pdf}
			\end{subfigure}
			%\vskip .1cm
			\begin{subfigure}{.5\linewidth}
				\centering
				\includegraphicsw[.7]{KH_clouds_1.jpg}
			\end{subfigure}%
			\begin{subfigure}{.5\linewidth}
				\small{Finest solution loses ${\sim0.5\%}$ of kinetic energy when the flow is dominated by two counter-rotating vortexes; this compares well with results computed with a higher order method in~[5] for planar case}
			\end{subfigure}
		\end{figure}
		\vfill
		\setul{1pt}{.4pt} % https://tex.stackexchange.com/questions/50637/closer-underline
		\tiny{
			\textsuperscript{5}P.~Schroeder, V.~John, P.~Lederer et al.\\
			$\:\:$\href{https://www.sciencedirect.com/science/article/pii/S0898122118306278}{\ul{On reference solutions and the sensitivity of the 2D Kelvin--Helmholtz instability problem}},\\ $\:\:$Computers \& Mathematics with Applications 2019
		}
	\end{frame}

%	
%	\section{Theoretical background}

%	\begin{frame}{Finite element method workflow \& motivation}
%		\begin{enumerate}
%			\item $L\,u = f$ + BCs $\Rightarrow$ Time discretization, linearization and finite element approximation $\Rightarrow$ Linear system~$\vect A\,\vect x = \vect b$ with \textbf{stiffness matrix}~$\vect A \in \mathbb R^{n\times n}$
%			\item For~$L = -\nabla\cdot(\vect K\,\nabla)$ we have~$\vect A_{ij} = \int \vect K\,\nabla\phi_j\cdot\nabla\phi_i$ with $\phi_i$ being a basis function with a compact support; $\phi_i$ is typically chosen as a piecewise polynomial of degree~$k$.
%		\end{enumerate}
%		\begin{figure}
%			\begin{subfigure}{.5\linewidth}
%				\centering
%				\includegraphicsw[.8]{sparse.pdf}
%				{Nonzero pattern of~$\vect A$}
%			\end{subfigure}%
%			\begin{subfigure}{.5\linewidth}
%				$\vect A$ is \textbf{sparse}, \\i.e. it requires~$O(n)$ memory resources and $O(n)$ FLOPs when multiplying the matrix by a vector. 
%			\end{subfigure}
%		\end{figure}
%	\end{frame}
%
%	\begin{frame}{Stiffness matrix assembly}
%		\begin{itemize}
%		\item For large complex problems, especially in 3D, it is typical to use an iterative solver with a suitable preconditioner to solve the system
%		\item The core operation is~$\vect x \mapsto \vect A\,\vect x \eqqcolon \vect z$
%		\item Hence it is crucially important that the sparse matrix-vector multiplication is implemented efficiently.
%		\end{itemize}
%		\begin{figure}
%			\begin{subfigure}{.45\linewidth}
%				\centering
%				\includegraphicsw[.8]{mesh.png}
%				Domain mesh
%			\end{subfigure}%
%			\begin{subfigure}{.55\linewidth}
%				$\vect A$ is assembled from small cell matrices~$\vect A_\text{cell} \in \mathbb R^{k^{\text{dim}}\times k^{\text{dim}}}$. Formally it can be written as
%				$$
%					\vect A = \sum \vect P^T_\text{cell}\,\vect A_\text{cell}\,\vect P_\text{cell},
%				$$
%				where rectangular matrix~$\vect P_\text{cell}$ connect local cell d.o.f. to global mesh ones.
%			\end{subfigure}
%		\end{figure}
%	\end{frame}	
%
%	\begin{frame}[fragile]{Sparse matrix-vector multiplication}
%		It is very common that $\vect A$ is represented in CSR format. In this case multiplication takes the form:
%		\begin{lstlisting}[basicstyle=\small]
%			for (i = 0; i < nrows; ++i) 
%				for (k = rowptr[i]; k < rowptr[i+1]; ++k)
%					z[i] += val[k] * x[colind[k]];
%		\end{lstlisting}
%		CSR is optimal in the sense that both storage and multiplication require~$O(n)$ resources. However,
%		\begin{enumerate}
%			\item there is no spatial locality for~$\vect x$, and
%			\item there are 5 memory refs vs. 2 arithmetic operations, i.e. the routine is memory bandwidth bounded
%		\end{enumerate}
%	\end{frame}	
%	
%	\begin{frame}{Matrix-free approach: Idea}
%	The alternative approach is \textbf{matrix-free} implementation: Do not store the big sparse matrix~$\vect A$ but compute the action of its cell matrices~$\vect A_\text{cell}$ on the fly.
%		\begin{enumerate}
%			\item Low arithmetic intensity is the bottleneck for matrix-based computations; Matrix-free evaluation that reads less data can be advantageous even if it does more computations
%			\item Matrix-free methods have a better complexity per degree of freedom:~$O(k)$ vs.~$O(k^\text{dim})$ for matrix-based methods ($k$ is basis func degree); This makes it attractive for higher order elements $k > 1$ and dim = 3~\includegraphics[width=12px]{clown.png}
%		\end{enumerate}
%	\end{frame}	
%	
%	\begin{frame}{Matrix-free approach: Implementation}
%		Our aim is to rewrite
%		$$
%			\vect x \mapsto \vect A\,\vect x = \big(\sum \vect P^T_\text{cell}\,\vect A_\text{cell}\,\vect P_\text{cell}\big)\vect x = \sum \vect P^T_\text{cell}\,\vect A_\text{cell}\,\vect x_\text{cell}
%		$$
%		w/o storing local matrices~$\vect A_\text{cell}$. Note that
%		$$
%			\vect A_{\text{cell}\,ij} = \int_\text{ref\_cell} \vect K_\text{ref\_cell}\,(\vect J^{-T}_\text{cell}\,\nabla\phi_{\text{ref\_cell}\,j})\cdot(\vect J^{-T}_\text{cell}\,\nabla\phi_{\text{ref\_cell}\,i})\,|\text{det}\,
%			\vect J|
%		$$
%		can be rewritten as
%		$$
%			\vect A_\text{cell}\,\vect x_\text{cell} = \vect B^{T}_\text{ref\_cell}\,\vect J^{-1}_\text{cell}\,\vect D_\text{cell}\,\vect J^{-T}_\text{cell}\,\vect B_\text{ref\_cell}\,\vect x_\text{cell}
%		$$
%		with \textbf{fixed} gradient mtx~$\vect B_\text{ref\_cell}$, reference-to-physical cell transformation~$\vect J_\text{cell}$, and diagonal diffusion mtx~$D_\text{cell}$. This gives us
%		\begin{itemize}
%			\item computation of~$\vect A_\text{cell}$ on the fly via applying matrix-vector products repeatedly, and
%			\item $O(k)$ complexity per degree of freedom
%		\end{itemize}
%	\end{frame}	
%
%	\begin{frame}{Execution time}
%		We stick to a 3D diffusion problem, $L = -\nabla\cdot(\vect K\,\nabla)$, with~$k = 3$. Our measurements are based on 5 matrix-vector multiplications per run on \href{https://portal.tacc.utexas.edu/user-guides/stampede2}{TAAC} skx-normal partition node (2 sockets, 48 cores, 96 threads)
%		\begin{figure}
%			\centering
%			\begin{subfigure}{.5\linewidth}
%				\centering\small{CSR}
%				\includegraphicsw{clTime.pdf}
%			\end{subfigure}%
%			\begin{subfigure}{.5\linewidth}
%				\centering\small{Matrix-free}
%				\includegraphicsw{mfTime.pdf}
%			\end{subfigure}%
%			\vskip .5cm
%			\begin{subfigure}{.8\linewidth}
%				\centering\small{Best thread team sizes}
%				\vskip .3cm
%				\includegraphicsw{bestTime.pdf}
%			\end{subfigure}
%		\end{figure}
%	\end{frame}
%
%	\begin{frame}{Other statistics}
%		\begin{figure}
%			\centering\small{Cache miss ratio (left) and instructions per cycle (right) for the largest matrix size~$n = 912\,673$}
%			\vskip .3cm
%			\begin{subfigure}{.5\linewidth}
%				\includegraphicsw{cache.pdf}
%			\end{subfigure}%
%			\begin{subfigure}{.5\linewidth}
%				\includegraphicsw{inst.pdf}
%			\end{subfigure}%
%			\vskip .5cm
%			\begin{subfigure}{.8\linewidth}
%				\centering\small{Energy consumption for a single thread; TDP = 150\,W}
%				\vskip .3cm
%				\includegraphicsw{rapl.pdf}
%			\end{subfigure}
%		\end{figure}
%	\end{frame}	
%	
%	\begin{frame}{Summary}
%		Matrix-free vs. CSR matrix-vector multiplication approaches: 
%		\begin{itemize}
%			\item We see that the matrix-free approach scales well as one increases the number of threads, and CSR does not; Moreover,
%			\item its inst/cycle ratio decays linearly (not the case for CSR!), and \% of cache misses decreases from 31 to 20 (vs. 41 to 34 for CSR). For the largest mtx size its execution time turns out to be $>2$ times better than ditto for CSR approach.
%			\item The energy consumption is quite comparable; Matrix-free approach spends slightly more energy on CPU than CSR approach.
%			\item This confirms that the matrix-free approach is much less memory bandwidth bounded than CSR.
%			\item Some drawbacks: Constructing matrix-free preconditioners is not straightforward; Implementation for more complicated stiffness matrices is not straightforward
%		\end{itemize}
%	\end{frame}
	
\end{document}


